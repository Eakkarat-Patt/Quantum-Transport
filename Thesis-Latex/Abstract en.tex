\noindent 
{\begin{tabular}{ll} 
  Thesis Title &  Electronic Transport of Dirac Fermion in Tilted Velocity \\
               &  Modulated Dirac Material Junction \\
  Thesis Credits & 12 \\
  % Thesis Registration Credits & 1 \\
  Candidate & Mr. Eakkarat Pattrawutthiwong \\
  Thesis Advisor & Asst. Prof. Watchara Liewrian \\
  Program & Master of Science  \\
  Field of Study & Physics \\
  Department & Physics \\
  Faculty & Science \\
  Academic Year & 2020 \\
\end{tabular}}

\vspace{1cm}

\centerline{Abstract}

\vspace{1cm}

The tilt-mismatch effect on resonant tunneling through an electrical potential barrier in asymmetric tilt-Dirac cone junction is investigated.
By varying barrier height, the angle-selective transmission of resonant tunneling oscillates as a function of gate voltage, and the linear phase shift due to the increase in the tilted parameter.
We found that the signature of the tilt parameter can be determined by measuring the tunneling transport properties across the tilt-mismatch junction.
For tilt-homogeneous junction, the tilt-induced pseudo-magnetic effect can occur only when an electric potential is applied to the system.
However, the asymmetric tilt-energy dispersion systems also can mimic the pseudo-magnetic barrier structure without the electric potential.
This result opens the opportunity for the tilt-Dirac cone system's magnetic focusing applications in electron-optics without magnetism.

\vspace{1cm}


{Keywords\;:}  Tilted Dirac cone/ Pseudo-magnetic effect/ Quantum transport


  
