\kmuttchapter{INTRODUCTION}
\section{Background and motivation}
    Electronic transport in a two-dimensional (2D) system has been a popular topic in condensed matter physics since the first rise of Dirac material in 2004, known as graphene \cite{Zhang2004,Wehling2014}. 
    Their novel transport properties arise from linear energy dispersion, where charge carriers mimic massless Dirac fermion (MDF) \cite{CastroNeto2009}. 
    Graphene is one example of isotropic Dirac cone material, where energy dispersion around Dirac point has the same slope in $k_x$ and $k_y$ directions. 
    However, theoretical studies revealed that graphene Dirac cone could exist in the tilted and anisotropic manner by band engineering. 
    For example, by applying the periodic potential to graphene sheet, the anisotropy of Dirac cone can be tuned \cite{Park2008}. 
    The DFT calculation predicted that hydrogenated graphene exhibits Dirac material with tilted anisotropic Dirac cones \cite{Lu2016}. 
    The nitrogen line defects in graphene are also predicted to induce type-II over-tilted Dirac cone \cite{Zhang2017a}.\\

    In the last decade, great attention has been paid to investigate for new Dirac materials with anisotropic electronic properties. 
    For example, two-dimensional stacked layers of phosphorene known as black phosphorus \cite{Xia2014,Kim2015} and bulk structure of $\mathrm{SrMnBi_2}$ \cite{Park2011}. 
    Recently, borophene, a 2D allotrope of boron, has successfully grown on silver surfaces and predicted to host anisotropic tilted MDF \cite{Mannix2015,Zhou2014}. 
    Also, it’s been reported that the anisotropy of high-Tc cuprate superconductors can be modulated by the hole doping \cite{Marino2019}.
    Anisotropic and tilt of Dirac material offer some novel electronic properties; for example, the superconducting gap has been predicted to be enhanced during the phase transition between type-I and type-II cone in Weyl semimetal \cite{Li2017a}. 
    In the context of transport properties, it has been demonstrated that electron optic behaviors with the opposite chiralities refracted into opposite directions, which may be useful for valley filtering \cite{Nguyen2018a}. 
    The tilted strength of a Weyl semimetal combined with the magnetic field effect considerably enhances the conductance gap \cite{Yesilyurt2017}. 
    The potential barrier and tilt effect is also predicted to induce the pseudo-magnetic field resulting in the asymmetric transmission \cite{Yesilyurt2017a}. 
    Therefore, it is beneficial to identify the tilted signatures of the Dirac cone to fine-tuning the electronic properties. 
    In fact, It has been shown that the anisotropic tilted Dirac cone in $\mathrm{\alpha-(BEDT-TTF)_2I_3}$ organic compounds can be measured by analyzing interlayer magnetoresistance \cite{Morinari2009}. 
    Recently, It has been also demonstrated that the Fano factor is sensitive to the tilt of the cone and can be used to verify the tilted signatures of material \cite{Trescher2015}.\\

    In this work, the tunneling properties of electrons across the mismatch of the tilted Dirac cone are investigated. 
    We first study the effect of the mismatch of the tilted Dirac cone on the electron transmission or resonance transmission in particular. 
    We then analyze the tunneling resonance condition of electron and propose a method to measure the tilted strength of the Dirac cone. 
    Moreover, we study the transmission under the influence of pseudo-magnetic field induced by the mismatch of the tilted Dirac cones

    

    %The electronic properties of material are defined by the ability of charge carriers to move throughout the crystal structures
    %The ability of charge carriers to tunnel throughout the material is defines by the crystal, electronic structures of their host material.
    %The electronic structure of material is critical, it defines the the ability of charge carriers to tunnel across the material.
    %Is has been proven that
    %The tunneling properties

