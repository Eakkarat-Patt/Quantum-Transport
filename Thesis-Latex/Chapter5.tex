\kmuttchapter{CONCLUSION}
    %the effect of the tilt does not destroy the uniformity of the oscillation, it simply shifted as the tilted parameter increased.
    %The transmission probabilities of electron across tilted Dirac cone undergo the effect of pseudo magnetic field
    The observation of electronic structure has been often performed by angle-resolved photoemission spectroscopy (ARPES), which is the standard method to investigate electronic properties of material. 
    It has been suggested previously that the novel transport properties arise from anisotropy and tilt may be used to indirectly determine the character of Dirac cone \cite{Zhang2018b}. 
    Here we have shown that the asymmetric tunneling resonance can be utilized to verify the existence of the tilt. 
    By tuning the appropriate voltage at the top gate, the tilted parameter can be determined. 
    Which may be useful as an alternative method to study the electronic structure of materials. 
    We also show that the electron transport behaviors across non-tilted/tilted/non-tilted heterostructure mimic the particle under the influence of real-magnetic field in non-tilted Dirac cone system. 
    This study may be utilized for magnetic confinement applications such as magnetic waveguide. 
    This device has been previously proposed where the stripes of ferromagnetic material are used to generate the MVP barrier. 
    Which from the experimental point of view, implementing the ferromagnetic material into the device is quite challenging \cite{Awschalom2009}. 
    The pseudo-magnetic field may pave the way to design magnetic devices without magnetism. 